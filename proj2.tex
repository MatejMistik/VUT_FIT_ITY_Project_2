\documentclass[a4 paper,11pt]{article}
\usepackage[utf8]{inputenc}
\usepackage[czech]{babel}
\usepackage[T1]{fontenc}
\usepackage[IL2]{fontenc}
\usepackage[unicode]{hyperref}
\usepackage[left=1.5cm,text={18cm, 25cm},top=2.5cm]{geometry}
\usepackage{amsfonts}
\usepackage{mathtools}
\usepackage{amsthm}
\usepackage{times}
\usepackage{amsmath}
\usepackage{amssymb}

\theoremstyle{definition}
\newtheorem{definice}{Definice}
\theoremstyle{plain}
\newtheorem{veta}{Věta}



\begin{document}

\begin{titlepage}
\begin{center}

\huge
{FAKULTA INFORMAČNÍCH TECHNOLOHIÍ\\[0.4em]
VYSOKÉ UČENÍ TECHNICKÉ V BRNĚ}\\
\vspace{\stretch{0.382}}
\LARGE
{Typografie a publikování -- 2. projekt \\[0.3em] 
Sazba dokumentů a matematických výrazů}
\vspace{\stretch{0.618}}

\textbf{{\LARGE 2020 \hfill 
{Matej Mištík (xmisti00)}}}

\end{center}
\end{titlepage}


\twocolumn

 \section*{Úvod}
 
V této úloze si vyzkoušíme sazbu titulní strany, matematických vzorců, prostředí a dalších textových struktur obvyklých pro technicky zaměřené texty (například rovnice (2) nebo Definice 2 na straně 1). Pro vytvoření těchto odkazů používáme příkazy \verb|\label|, \verb|\ref| a \verb|\pageref|. \\
Na titulní straně je využito sázení nadpisu podle optického středu s využitím zlatého řezu. Tento postup byl probírán na přednášce. Dále je použito odřádkování se zadanou relativní velikostí 0.4em a 0.3em.
	


\section{Matematický text}
Nejprve se podíváme na sázení matematických symbolů a výrazů v plynulém textu včetně sazby definic a vět s využitím balíku amsthm. Rovněž použijeme poznámku pod čarou s použitím příkazu \verb|\footnote.| Někdy je vhodné použít konstrukci \verb|${}$| nebo \verb|\mbox{}| která říká, že (matematický) text nemá být zalomen. V následující definici je nastavena mezera mezi jednotlivými položkami \verb|\item| na 0.05em.


    
\begin{definice} Turingův stroj (TS) \textit {je definovám jako šestice tvaru} $M = \left(Q,\sum, \Gamma, q_{0}, q_{F} \right)$, kde:

\begin{itemize}
\setlength{\itemsep}{0.05em}

 \item $Q$ \textit{je konečná množina}  vnitřních (řídících) stavů,
 
 \item $\sum$ \emph{je konečná množina symbolů nazývaná} vstupní abeceda, $\Delta \notin \sum $
 
 \item $\Gamma$ \emph{je konečná množina symbolů},$\sum \subset \Gamma, \Delta \in \Gamma $ , \emph nazývaná pásková abeceda,

 
 \item $\delta:\left(Q \backslash \left\{q_{F} \right\} \right) \times \Gamma \rightarrow Q \times(\Gamma \cup\{L,R\})$,\emph{kde}$L,R\notin \Gamma,$\emph{ je parciální} přechodová funkce, \emph{a}
 
 \item $q_{0} \in Q $ \emph je počáteční stav a $q_{f} \in Q $ \emph je koncový stav.
 
 
\end{itemize}

Symbol $\Delta$ značí tzv. \emph blank (prázdný symbol), který se vyskytuje na místech pásky, která nebyla ještě použita.
	
	Konfigurace pásky se skládá z nekonečného řetězce, který reprezentuje obsah pásky a pozice hlavy na tomto řetězci. Jedná se o prvek množiny $\left\{\gamma \Delta^{\omega} | \gamma \in \Gamma^{*}\right\} \times \mathbb{N}$\footnote{Pro libovolnou abecedu $\sum$ je $\sum^{\omega}$ množina všech nekonečných řetezců nad $\sum$, tj. nekonečných posloupností symobolů ze $\sum$} . Konfigurační pásky obvykle zapisujeme jako $\Delta x y z \underline{z} x \Delta ...$ (podtržení značí pozici hlavy). Konfigurace stroje je pak dána stavem řízení a konfigurační pásky. Formálně se jedná o prvek množiny $Q \times\left\{\gamma \Delta^{\omega} | \gamma \in \Gamma^{*}\right\} \times \mathbb{N}$.

\end{definice} 






\subsection{Podsekce}

\begin{definice} Řetězec $w$ nad abecedou $\sum$ je přijat \large TS $M$ jestliže $M$ při aktivaci z počáteční konfigurace pásky  $\Delta w \Delta \ldots$ a počátečního stavu $q_{0}$ zastaví přechodem do koncového stavu $q_{F}$, tj. $\left(q_{0}, \Delta w \Delta^{\omega}, 0\right) \underset{M}{\overset{*}{\vdash}} \left(q_{F}, \gamma, n\right)$ pro nějaké $\gamma \in \Gamma^{*} a n \in \mathbb{N}$.

Množinu $L(M)=\{w | w \text { \emph je přijat TS } M\} \subseteq \Sigma^{*}$ nazýváme jazyk přijímaný TS M.

Nyní si vyzkoušíme sazbu vět a důkazů opět s použitím balíku \verb|amsthm|.

\begin{veta} Třída jazyků, které jsou přijímány TS, odpovídárekurzivně vyčíslitelným jazykům.

\end{veta}

\begin{flushleft}

\itshape Důkaz. V důkaze vyjdeme z Definice 1 a 2. \hfill $\square$

\end{flushleft}

\end{definice}



\section{Rovnice}

Složitější matematické formulace sázíme mimo plynulý text. Lze umístit několik výrazů na jeden řádek, ale pak je třeba tyto vhodně oddělit, například příkazem \verb|\quad|.

\vspace{0.618cm}


\quad{$\sqrt[i]{x_{i}^{3}} \quad \text { kde } x_{i} \text { je } i \text { i-té sudé číslo}\quad y_{i}^{2 \cdot y_{i}} \neq y_{i}^{y_{i}^{y_{i}}}$}    
 
\vspace{0.382cm}


V rovnici (1) jsou vužity tři typy závorek s různou explicitně definovanou velikostí.

\begin{quote}


\hspace{1.cm}$x \quad= \quad\left\{([a+b] * c)^{d} \oplus 1\right\}$ \hfill (1)

\hspace{0.9cm} $y \quad=\quad  \lim \limits_{x \to \infty} \frac{\sin ^{2} x+\cos ^{2} x}{\frac{1}{\log _{10} x}}$ \hfill (2)


\end{quote}

V této větě vidíme, jak vypadá implicitní vysázení limity $\lim_{n \to \infty} f(n)$ v normálním odstavci textu. Podobně je to i s dalšími symboly jako $\sum_{i=1}^{n} 2^{i} \, \text {či} \,  \bigcap_{A \in \mathcal{B}} A$. V případě vzorců $\lim\limits _{n \rightarrow \infty} f(n)$ a $\sum\limits_{i=1}^n 2^i$ jsme si vynutili méně úspornou sazbu příkazem \verb|\limits|.


\section{Matice}

Pro sázení matic se velmi často používá prostředí array a závorky (\verb|\left, \right|).


\begin{quote}
    


\quad
$\left(\begin{array}{ccc}
a+b & \widetilde{\xi+\omega} & \hat{\pi} \\
\vec{\mathbf{a}} & \stackrel{\longrightarrow}{A C} & \beta
\end{array}\right)=1 \Longleftrightarrow \mathbb{Q}=\mathcal{R}$

\end{quote}

Prostředí \verb|array| lze úspěšně využít i jinde.





\vspace{0.618cm}

    
\quad$\left(\begin{array}{l}
n \\
k
\end{array}\right)=\left\{\begin{array}{cl}
0 & \text { pro } k<0 \text { nebo } k>n \\
\frac{n !}{k !(n-k) !} & \text { pro } 0 \leq k \leq n
\end{array}\right.$

    
\vspace{0.382cm}

\end{document}

